\documentclass[12pt,a4paper,dvipdfmx]{jsarticle}

\usepackage[utf8]{inputenc}     % 文字コードをUTF-8に指定
\usepackage{graphicx} 					% グラフィクスの利用宣言 (変更: dvipdfmxはdocumentclassで指定)
\usepackage{multirow}           % 表でのセル結合
\usepackage{dcolumn}            % 表での小数点揃え
\usepackage{amssymb}            % 数学記号などの利用宣言
\usepackage{mathtools}          % 数式の記述を助けるパッケージ
\usepackage{amsmath}            % 数式の記述を助けるパッケージ
\usepackage{siunitx}            % SI単位系の利用宣言
\usepackage{bm}                 % 太字数式文字の利用に関する宣言
\usepackage{float}              % 図表の位置を固定する
\usepackage{url}                % URLの記述を助けるパッケージ
\usepackage{listings}           % ソースコードの記述を助けるパッケージ
\usepackage{jvlisting}          % ソースコードの日本語対応
\usepackage{xcolor}             % 色の指定
\usepackage{tikz}               % 図の描画を助けるパッケージ

%%%%%%%%%%%%%%%%%%%%%%%%%%%%%%%%%%%%%%%%%%%%%%%%%%%%%%%%%%%%%%%%%%%%%%%%%%%%%%%%%%%%%%%%%%%%%%%%%%%%

\title{これはサンプルです}
\author{0000000 \quad ほげほげ}
\date{2000年00月00日}

%%%%%%%%%%%%%%%%%%%%%%%%%%%%%%%%%%%%%%%%%%%%%%%%%%%%%%%%%%%%%%%%%%%%%%%%%%%%%%%%%%%%%%%%%%%%%%%%%%%%

% ここから本体
\begin{document}

\pagenumbering{arabic}           % ページ番号をアラビア数字になおす
\renewcommand{\figurename}{図}    % キャプションをFigureから図にする
\renewcommand{\tablename}{表}     % キャプションをTableから表にする
\numberwithin{equation}{section} % 式番号をセクション毎にリセットする
\maketitle                       % タイトルを出力

%%%%%%%%%%%%%%%%%%%%%%%%%%%%%%%%%%%%%%%%%%%%% ↓ EDIT ↓ %%%%%%%%%%%%%%%%%%%%%%%%%%%%%%%%%%%%%%%%%%%%%

\section{タイトル}

ここに,$d \times n$ の行列 $\mathbf{X}$ がある (ただし $d \geq n$).
行列 $\mathbf{X}^{t}\mathbf{X}$ の固有値とと固有ベクトルをそれぞれ $\lambda_{i}, \mathbf{e}_{i} \quad (i = 1, 2, \ldots, n)$ とする.
このとき,行列 $\mathbf{X}\mathbf{X}^{t}$ の非ゼロな固有値とそれに対応する固有ベクトルは,それぞれ $\lambda_{i}, \frac{\mathbf{X}\mathbf{e}_{i}}{\sqrt{\lambda_{i}}} \quad (i = 1, 2, \ldots, d)$ となることを示す.

行列 $\mathbf{X}^{t}\mathbf{X}$ を $n \times n$ の行列 $\mathbf{A}$ とし,行列 $\mathbf{X}\mathbf{X}^{t}$ を $d \times d$ の行列 $\mathbf{B}$ とする.
行列 $\mathbf{A}$ の固有値と固有ベクトルは次のように定義される.
\begin{equation}
	\mathbf{A}\mathbf{e}_{i} = \lambda_{i}\mathbf{e}_{i} \quad (i = 1, 2, \ldots, n) \label{eq:7-1-1}
\end{equation}
ここで,行列 $\mathbf{A}$ は対称行列であるため,固有値は実数であり,固有ベクトルは直交する.
よって,固有ベクトル $\mathbf{e}_{i}$ が正規直交基底と仮定すると,
\begin{equation}
	\mathbf{e}_{i}^{t}\mathbf{e}_{j} = \begin{cases}
		1 & (i = j)    \\
		0 & (i \neq j)
	\end{cases} \quad (i, j = 1, 2, \ldots, n)
\end{equation}
が成り立つ.
ここで,式\eqref{eq:7-1-1}において,行列 $\mathbf{X}$ を両辺に左から掛けると,
\begin{equation}
	\mathbf{X}\mathbf{A}\mathbf{e}_{i} = \lambda_{i}\mathbf{X}\mathbf{e}_{i} \quad (i = 1, 2, \ldots, n)
\end{equation}
となる.
$\mathbf{A} = \mathbf{X}^{t}\mathbf{X}$ を代入すると,
\begin{equation}
	\mathbf{X}\mathbf{X}^{t}\mathbf{X}\mathbf{e}_{i} = \lambda_{i}\mathbf{X}\mathbf{e}_{i} \quad (i = 1, 2, \ldots, n).
\end{equation}
ここで,行列 $\mathbf{X}\mathbf{X}^{t}$ は行列 $\mathbf{B}$ であるため,
\begin{equation}
	\mathbf{B}\mathbf{X}\mathbf{e}_{i} = \lambda_{i}\mathbf{X}\mathbf{e}_{i} \quad (i = 1, 2, \ldots, n).
\end{equation}
この式は,行列 $\mathbf{B}$ の固有値と固有ベクトルの定義に一致する.
したがって,行列 $\mathbf{B}$ の固有値は $\lambda_{i}$ であり,対応する固有ベクトルは $\mathbf{X}\mathbf{e}_{i}$ である.
さらに,$\mathbf{X}\mathbf{e}_{i}$ のノルムを計算すると,
\begin{equation}
	\|\mathbf{X}\mathbf{e}_{i}\| = \sqrt{(\mathbf{X}\mathbf{e}_{i})^{t}(\mathbf{X}\mathbf{e}_{i})} = \sqrt{\mathbf{e}_{i}^{t}\mathbf{X}^{t}\mathbf{X}\mathbf{e}_{i}} = \sqrt{\lambda_{i}}.
\end{equation}
したがって,行列 $\mathbf{B}$ の非ゼロな固有値は $\lambda_{i}$ であり,対応する固有ベクトルは
\begin{equation}
	\frac{\mathbf{X}\mathbf{e}_{i}}{\|\mathbf{X}\mathbf{e}_{i}\|} = \frac{\mathbf{X}\mathbf{e}_{i}}{\sqrt{\lambda_{i}}} \quad (i = 1, 2, \ldots, n)
\end{equation}
となる.

\subsection{サブタイトル}

論文やレポートでは,数式を本文中に組み込むことがよくあります.
オイラーの等式 $e^{i\pi} + 1 = 0$ は,数学における最も美しい方程式の一つとされています.\\
\indent
論文の引用は,例えば \text{$\setminus$ cite\{ref\}} のように行います\cite{me}.

%%%%%%%%%%%%%%%%%%%%%%%%%%%%%%%%%%%%%%%%%%%%% ↑ EDIT ↑ %%%%%%%%%%%%%%%%%%%%%%%%%%%%%%%%%%%%%%%%%%%%%

% \clearpage

\bibliographystyle{junsrt} % 参考文献スタイルを指定
\bibliography{ref} % 参考文献 ref.bib ファイルを参照

\end{document}